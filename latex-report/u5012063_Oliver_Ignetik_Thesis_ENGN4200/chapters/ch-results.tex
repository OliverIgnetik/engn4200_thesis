\SetPicSubDir{ch-Results}
\SetExpSubDir{ch-Results}

\chapter{Results}
\label{ch:results}
\vspace{2em}

\section{NMF approach to monophonic AMT}

\subsection{Discussion}

In this section the results obtained by the \ac{NMF} model will be discussed.
The four most influential parameters tested in the model will be discussed in this section.


\begin{table}[!ht]
    \centering
    \begin{tabular}{|l|c|}
        \hline
        \textbf{Parameters}       & \multicolumn{1}{l|}{\textbf{Value}} \\ \hline
        Window size               & 3000                                \\ \hline
        Threshold value           & 0.5                                 \\ \hline
        harmonics                 & 5                                   \\ \hline
        Attack frames             & 0                                   \\ \hline
        \textbf{Overall Accuracy} & \multicolumn{1}{l|}{\textbf{1.00}}  \\ \hline
    \end{tabular}
    \caption{Optimal parameters for NMF model}
    \label{results:table:nmf-optimal}
\end{table}

\begin{figure}
    \centering
    \includegraphics[width=0.98\columnwidth]{\Pic{png}{NMF_tuning}}
    \caption[Hyperparameter gridsearch]{Hyperparameter tuning with accuracy metrics for monophonic \ac{AMT} system}
    \label{review:fig:NMF-tuning}
\end{figure}

\subsubsection{Window Size}
Through inspecting \autoref{review:fig:NMF-tuning}~(a) it is shown that the
overall accuracy of the model decreased as the window size increased. This can
be explained by referring back to \autoref{review:eq:tradeoff} which
encapsulates the concept of time-frequency resolution trade off in spectrograms.
The ideal time window size to use in this situation is about 3004 samples as this
translates to a frequency resolution of 14.68~$\si{\hertz}$. This approximately equivalent
to a semi-tone for the mid-range of the piano as shown in \autoref{review:eq:equal-scale}.

\subsubsection{Peak Picking Threshold Value}
One of the crucial parameters in the model is the use of a peak picking
algorithm that depends on a number of parameters. Most notable of these
parameters is the delta threshold value which translates to the sensitivity of
the model to ambient noise. The NMF model returns a dictionary matrix and an
activation matrix. The dictionary matrix (example shown in
\autoref{results:fig:dictionary}) can be thought of as \ac{DFT}s of the seperate
notes that together construct the final audio signal. The threshold function
accounts for ambient noise and is context dependent on the recording
environment. This means for each sample recording and differing instrument the
threshold function has to be tuned. It can be seen in
\autoref{review:fig:NMF-tuning}~(b) that by increasing the threshold the overall
accuracy decreases. Thus the optimal range for this recording environment is
approximately 0.5.

\begin{figure}
    \centering
    \includegraphics[width=0.98\columnwidth]{\Pic{png}{dictionary}}
    \caption{Example of dictionary components}
    \label{results:fig:dictionary}
\end{figure}

\subsubsection{Number of harmonics}
The number of harmonics is another crucial parameter in finding the fundamental
frequency. As discussed in \autoref{subsection:pitch-harmony} a pitched sound
comprises of a series of harmonics that are integer multiples of the fundamental
frequency. In fact only a number of harmonics are needed for the pitch to be
percieved in the mind of the listener. By inspecting
\autoref{review:fig:NMF-tuning}~(c), it is evident that only a small number of
harmonics are needed for good performance in the model. However, it is noted
that as the number of harmonics included in the calculation of the fundamental
frequency has an inverse relationship with overall accuracy. This can be
explained by a concept known as inharmonicity. This phenomenon is present in
many pitched instruments where the harmonics are not perfect integer multiples
of the fundamental frequency due to properties of the instrument that vary of
time such as string tension and hammer velocity in pianos.

\subsubsection{Number of attack frames}
The last important parameter that will be discussed in this section is the
number of attack frames. This parameter is important for predicting the onset
time of each note event. \autoref{results:fig:activations} is the activation
matrix showing the temporal evolution of the notes. Typical pitched events have four phases
that musicians would be well aware of attack, decay, sustain and release. The pitched event
will be best percieved in the mind of the listener during the sustain period as a stable frequency of
oscillation will be achieved. This experiment, however, revealed that disregarding the attack frames
in fact increased the model performance as shown in \autoref{review:fig:NMF-tuning}~(d).
The number of attack frames also depends on the type of recording environment and the type of instrument used.
So given a different instrument the number of attack frames may have to be changed.

\begin{figure}
    \centering
    \includegraphics[width=0.98\columnwidth]{\Pic{png}{activations}}
    \caption{Example of activation components}
    \label{results:fig:activations}
\end{figure}

\subsection{Summary}

In summary the \ac{NMF} model performed extremely well for monophonic AMT however is not
appropriate for polyphonic transcription. The crucial points of discovery are:

\begin{itemize}
    \item \ac{NMF} model provides high accuracy metrics for monophonic music
    \item Fast compile time of \ac{STFT} and \ac{NMF} decomposition
    \item Needs fine tuning dependent on recording environment and instrument type
    \item The attack frames had a negligible effect on model performance
\end{itemize}


\section{NN approach to polyphonic AMT}

\subsection{Discussion}
In this section the results obtained by the \ac{NN} model will be discussed.
Generally the model performed well achieving a validation f-measure of 0.75 after
100 epochs performed on the training data. The most important insights will be
discussed including the Hyperparameter tuning, choice of loss function and
metric of performance. The optimal parameters are shown in \autoref{results:table:nn-optimal}.


\begin{table}[!ht]
    \centering
    \begin{tabular}{|l|c|}
        \hline
        \textbf{Parameters} & \multicolumn{1}{l|}{\textbf{Value}} \\ \hline
        Neurons per layer   & 512                                 \\ \hline
        Dropout Rate        & 0.1                                 \\ \hline
        Layers              & 3                                   \\ \hline
        \textbf{f-measure}  & \multicolumn{1}{l|}{\textbf{0.75}}  \\ \hline
    \end{tabular}
    \caption{Optimal parameters for model}
    \label{results:table:nn-optimal}
\end{table}

\subsubsection{Hyperparameter grid search}
In order to investigate the effect of the model parameters on performance a
number of parameters were investigated as shown in
\autoref{results:table:nn-grid}. As the number of layers or number of neurons
goes up the compile and training time of the model goes up so this must be
carefully considered when using NN models. The dropout rate is important to set
to a value of 0.1 to ensure that the model does not overfit to the training
data. The ultimate goal of the model is to apply it to "unseen data" to fully
test its predictive power. This means the model is highly dependent to genre,
recording environment and instrumentation. This model would not perform well on
a music database that is vastly different as it would need to be retrained which
takes considerably more time then using an NMF model which is one major
drawback. Training this model for 100 epochs to achieve the observed f-measure
had a relatively long time of completion.

\begin{figure}
    \centering
    \includegraphics[width=0.98\columnwidth]{\Pic{png}{NN_perf}}
    \caption[NN model performance]{Loss function and f-measure curves for neural network model with optimal parameters}
    \label{results:fig:model-perf}
\end{figure}

\subsubsection{Loss function and metric of performance}
The behaviour of the curves in \autoref{results:fig:model-perf} indicates that
the use of the weighted cross entropy function in combination with the modified
f-measure are well suited to this type of problem. Initially a regular binary
cross entropy with binary accuracy was used however the model performed
poorly in that it overfit to inactive notes and provided a false sense of
accuracy due to this class imbalance.

\begin{table}
    \centering
    \begin{tabular}{|l|c|c|}
        \hline
                                          & \multicolumn{2}{c|}{\textbf{Validation scores}}                                            \\ \hline
        \textbf{Parameter}                & \multicolumn{1}{l|}{\textbf{f-measure}}         & \multicolumn{1}{l|}{\textbf{loss score}} \\ \hline
        \textbf{layers}                   & \multicolumn{1}{l|}{}                           & \multicolumn{1}{l|}{}                    \\ \hline
        2                                 & 0.374                                           & 0.374                                    \\ \hline
        3                                 & 0.382                                           & 0.368                                    \\ \hline
        4                                 & 0.370                                           & 0.378                                    \\ \hline
        \textbf{Dropout Rate}             & \multicolumn{1}{l|}{}                           & \multicolumn{1}{l|}{}                    \\ \hline
        0.0                               & 0.444                                           & 0.311                                    \\ \hline
        0.1                               & 0.382                                           & 0.369                                    \\ \hline
        0.2                               & 0.350                                           & 0.406                                    \\ \hline
        \textbf{Neurons in hidden layers} & \multicolumn{1}{l|}{}                           & \multicolumn{1}{l|}{}                    \\ \hline
        128                               & 0.359                                           & 0.398                                    \\ \hline
        256                               & 0.381                                           & 0.371                                    \\ \hline
        512                               & 0.399                                           & 0.346                                    \\ \hline
    \end{tabular}
    \caption[Hyperparameter gridsearch NN]{Hyperparameter tuning for \ac{NN} approach on polyphonic \ac{AMT}}
    \label{results:table:nn-grid}
\end{table}


\subsection{Summary}
In summary, the \ac{NN} model performed well for polyphonic AMT however further work
is needed to convert the output to an interpretible form that more closely resembles the piano \ac{MIDI} roll representation.
As it stands the results are still considered a mid-level representation that is far from
resembling a musical score which is the ultimate goal of AMT systems.
The crucial points of discovery are:
\begin{itemize}
    \item Stable loss curve illustrates use of proper accuracy metrics and loss functions
    \item f-measure was suitable for polyphonic music case
    \item Takes time to retrain data
    \item Output is not yet in piano \ac{MIDI} form
\end{itemize}




