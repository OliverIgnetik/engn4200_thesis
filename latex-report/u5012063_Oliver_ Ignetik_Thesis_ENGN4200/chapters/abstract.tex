\begin{abstract}

    This research paper explores the concept of Automatic Music Transcription. A
    literature review is conducted to provide a concise overview of the subject,
    including state of the art methods and how they can be used to better
    improve user satisfaction of current systems.

    In particular, this paper explores the method known as Non-negative Matrix
    Factorization as applied to time-frequency representations of audio signals.
    The primary concept that will be reviewed to aid with understanding this
    technique is the Short Time Fourier Transform.

    A secondary avenue of exploration is machine learning algorithms and their
    application to Automatic Music Transcription systems. A preliminary review
    is provided to readily prepare the reader for the related discussions and
    insights uncovered in this investigation.

    The design and application of a monophonic Non-negative Matrix Factorization
    and a polyphonic Neural Network system are presented followed by a
    discussion of the results. Thereafter, a discussion is presented on how
    higher level musical knowledge incorporated into future models to improve
    their accuracy.


\end{abstract}
