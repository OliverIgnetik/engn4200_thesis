\begin{abstract}

    This research paper explores the concept of Automatic Music Transcription
    (AMT). A literature review is conducted to provide a concise overview of the
    subject, including state of the art methods and how they can be used to
    better improve user satisfaction of current systems.

    In particular, this paper explores the method known as Non-Negative Matrix
    Factorization as applied to time-frequency representations of audio signals.
    The primary concept that will be reviewed to aid with understanding this
    technique is the Short Time Fourier Transform.

    A secondary avenue of exploration is machine learning algorithms and their
    application to AMT systems. A preliminary review is provided to readily
    prepare the reader for the related discussions and insights uncovered in
    this investigation.

    Finally the design and methodology of a monophonic AMT system is presented.
    Thereafter, a discussion is presented on how higher level musical knowledge
    such as diatonic harmonies, time signatures and modulations can be
    incorporated into future models to improve their accuracy.


\end{abstract}
