\SetPicSubDir{ch-Review} \SetExpSubDir{ch-Review}

\chapter{Background}
\label{ch:review}
\vspace{2em}

\section{Musical Concepts}
\subsection{Pitch and Harmony}
\label{subsection:pitch-harmony}
The existence of sequences of sounds with well-defined fundamental periods is a
very common feature in music. Most musical instruments such as pianos, guitars,
flutes and trumpets are constrcuted to allow performers to produce sounds with
easily controlled fundamental periods and associated harmonics. Such a signal is
described as a harmonic series of sinusoids at multiples of the fundamental
frequency and results in the perception of a pitch in the mind of the listener.

Although different cultures have developed different musical conventions, a
common feature is the musical "scale", a set of discrete pitches that repeats
every octave. In contemporary western music an "equal tempered scale" is used,
which divides the octave into 12 steps on a logarithmic axis called
semitones.\cite{NUS-perceptual-features:Ye}

\begin{equation}
  P_{n} = P_{a}(\sqrt[12]{2})^{n-a}
  \label{review:eq:equal-scale}
\end{equation}

Where :
\begin{conditions}
  P_{n}     &  Query pitch  \\
  P_{a}     &  Reference pitch \\
\end{conditions}

In musical theory, the spacing in between these steps are known as semitones and
form musical intervals. Different combinations of notes that form intervals
result in different harmonic structures or "colours" known as chords. Consonant
intervals like a perfect fifth are made up of seven semitones and have a
frequency ratio of ${(2^{\frac{1}{12}})}^{7} = \frac{3}{2}$ sounding pleasant to
the ear. They share many harmonics and ubiquitous in western music. This is
partly the reason why transcription can be so difficult. The tritone is
considered dissonant and has a intervallic frequency ratio of
${(2^{\frac{1}{12}})}^{6} = \frac{45}{32}$. This interval sounds jarring to the
ear and is associated with musical tension. Tritones provide a harmonic spine
for the movement of groups of notes because they are so noticable to the
listener.

\ac{MIDI} is one of the most important tools
for musicians. It is a protocal that allows computers, musical instruments and
other hardware to communicate. It encodes an audio signal into an
multi-dimensional array which contains information about the pitch and
onset/offset times of notes. Of particular note is the \ac{MIDI} pitch which has the
formula below :

\begin{equation}
  d = 69  + 12\log_{2}(\frac{f_{0}}{12})
\end{equation}

Where :
\begin{conditions}
  f_{0}     &  fundamental frequency  \\
  d     &  \ac{MIDI} number \\
\end{conditions}

On a grand piano the lowest note A0 has a frequency of 27.50 $\si{\hertz}$ and
midi number of 21. The highest note C8 has a frequency of 4186.0 $\si{\hertz}$
and midi number 108.

\subsection{Tempo, Beat and Rhythm}

The musical aspects of tempo, beat and rhythm play a fundamental role. The
\emph{beat} can be described as a sequence of perceived pulses that are
regularly spaced in time and correspond to the pulse a human taps along when
listening to the the music.
\cite{beat-def:Lerdahl}

The strength or stress of the musical pulse and how it varies determines the
metrical signature of a piece of music. Notes are grouped in rhythmic units in
each bar according to the time signature.

\begin{figure}[!ht]
  \centering
  \includegraphics[width=.6\linewidth]{\Pic{png}{time-sig}}
  \caption[Piano score excerpt]{Excerpt from a piano arrangement for the tune Nearness of You with a common time signature of 4 quarter notes per bar}
  \label{review:fig:time-sig}
\end{figure}

The term \emph{tempo} refers to the rate of this pulse as is often denoted as
\emph{beats per minute} or \emph{bpm}. Musical pulses typically coincide with
note onsets or percussive events. In the context of \ac{AMT} this tasks constitutes
finding a \emph{novelty curve} known as onset detection.

\section{Signal Processing Techniques}
\subsection{Sampling Theorem}

The sampling theorem is a consequence of digitizing analogue signals. Sampling
an analogue signal stores quantized values of the amplitude of a continuous
signal at regular intervals determined by the sampling rate.

The sampling theorem says that to avoid higher frequency components aliasing as
lower frequencies components the following must be satisfied. Considering a
sampling frequency $F_{s}$ and Nyquist frequency $F_{N}$.

\begin{equation}
  F_{s} > 2\cdot F_{N}
\end{equation}

Where $F_{N}$ is the highest frequency expected in the signal. Frequently a sampling
rate of 44.1 $\si{\kilo\hertz}$ is used in audio recording because the range of
human hearing is from 20$\si{hertz}$-20$\si{\kilo\hertz}$.

\subsection{Discrete Fourier Transform}

Consider a finite-length sequence $x[n]$ of length N samples such that $x[n] =
  0$ outside the range $0 \leq  n \leq N - 1$. To each finite-length sequence of
length N, it is possible to associate a periodic sequence.

\begin{equation}
  \tilde{x}[n] =\sum_{r=-\infty}^{\infty} x[n-r N]
\end{equation}

This assumption is implied in the mathematics of the \ac{DTFT}, that the signal of interest is
periodic in nature even when it has a finite length. The \ac{DTFT} of such a signal is given by :

\begin{equation}
  \tilde{X}[\omega] =\sum_{n=-\infty}^{n=\infty} \tilde{x}[n] \exp^{-j \omega n }
\end{equation}

This sequence is itself periodic with a period $N$. The \ac{DFT} of the original signal finite length signal $x[n]$ can be found
by sampling $\tilde{X}$ at $\omega=\frac{2 \pi}{N}$ and only considering the
values of k within $0 \leq  k \leq N - 1$ :

\begin{equation}
  X[k] =\sum_{n=0}^{n=N-1} x[n] \exp^{-j \frac{2 \pi}{N} k }
  \label{review:eq:DFT}
\end{equation}

The \ac{DFT} is often implemented as the \ac{FFT} which reduces the
order of complexity to $O(N\log{N})$ by exploiting symmetries in the
transformation. \cite{OppenheimDSP} \autoref{review:eq:DFT} will be used
frequently throughout this project and extended upon in
\autoref{subsection:STFT}.

\subsection{Short-Time Fourier Transform}
\label{subsection:STFT}

As in other audio-related applications, the most popular tool for describing the
time-varying energy across different frequency bands is the \ac{STFT}, which,
when visualized as its magnitude, is known as the spectrogram.

Formally, let $x$ be a discrete-time signal obtained by uniform sampling a
waveform at a sampling rate $F_{s}$ Hz. Using a N-point tapered window $w$ (eg.
Hamming $w[n] = 0.5-0.46\cdot cos(\frac{2\pi n}{N})$ for
$n\in\left[0,N-1\right]$) and an overlap of half a window length we obtain the
STFT.

\begin{equation}
  X [m,k] = \sum_{n=0}^{N-1}w[n]\cdot x[n + m\cdot\frac{N}{2}]\cdot exp\{-j\frac{2\pi k n }{N}\}
\end{equation}

With $m\in\left[0,T-1\right]$ and $k\in\left[0,K-1\right]$. Here, $T$ determines
the number of frames , $K = \frac{N}{2}$ is the index of the last unique
frequency value as dictated by the Sampling Theorem. Thus $X[m,k]$ corresponds :

\begin{equation}
  \begin{aligned}
    f_{coeff}(k) & = \frac{k}{N} \cdot F_{s} \mathrlap{\qquad [\si{\hertz}]}   \\
    t_{frame}(m) & = t \cdot \frac{N}{2F_{s}} \mathrlap{\qquad [\si{\second}]}
  \end{aligned}
  \label{review:eq:tradeoff}
\end{equation}

$X[m,k]$ is complex-valued, with the phase depending on the alignment of each
short-time analysis window. Often it is only the amplitude $\mid X[m,k] \mid$
that is used.
\cite{OppenheimDSP}

\subsubsection{Log-Frequency Spectrogram}

Note that the Fourier coefficients of $X[m,k]$ are linearly spaced on the
frequency axis. Using suitable binning strategies, various approaches switch
over to a logarithmically spaced frequency axis, by using mel-frequency bands or
pitch bands as seen in \autoref{review:fig:log-STFT-example}. Keeping the linear
frequency axis puts greater emphasis on the high-frequency regions of the
signal, thus accentuating the aforementioned noise bursts visible as
high-frequency content. One simple yet important step often applied in the
processing of music signals, is referred to as logarithmic compression. Such a
compression not only accounts for the logarithmic nature that describes how
humans perceive sound but also balances out the dynamic range of the signal.
Some variations of the traditional \ac{STFT} include \ac{CQT} and \ac{MFCC}.
\cite{spma2011:Klapuri}

\begin{figure}[!ht]
  \centering
  \includegraphics[width=.8\columnwidth]{\Pic{png}{log-stft-example}}
  \caption[Example Spectrogram]{\ac{STFT} of a 10s excerpt from Blues in F - Bill Evans Trio recording }
  \label{review:fig:log-STFT-example}
\end{figure}

\subsubsection{\ac{CQT}}

\ac{CQT} has a number of advantages over \ac{STFT} in note identification since constant center
frequency-to-resolution ratio results in a constant pattern of  sounds with harmonic components in the
logarithm-scaled frequency domain, which is easier for resolving notes that are played simultaneously. Not to
mention the fact that it more closely resembles the human auditory system which makes it ideal in \ac{AMT} \cite{LiDL2019}.

\section{State of the Art Methods}

Many approaches have been developed for \ac{AMT} applied to polyphonic music. While
the end goal of \ac{AMT} is to convert an acoustic music recording to some form of
music notation, most approaches are aimed at achieving an intermediate goal.
Some commercial applications provide the capability of converting a piano-roll
representation into typeset music notation. However, the end results are
generally musically illogical especially in genres like jazz where notes often
fall on the upbeat and rythms are highly syncopated.

\ac{AMT} approaches can generally be organized into four categories : frame-level,
note level, stream level and notation level. Frame level transcription which is
also known as \emph{\ac{MPE}} aims at identifying the number and
pitch of notes that are present in a frame of music. A frame is generally on the
time scale of 10ms depending on the type of analysis window. Note-level
transcription not only estimates the pitch in each time frame but also the onset
and offset times. Stream level transcription of \emph{instrument tracking}
targets the grouping of estimated notes into streams. These groupings typically
correspond to different instruments or timbres. Notation level transcription or
\emph{audio-to-note transcription} aims to transcribe the music audio into a
musical score such as that seen on staff notation. Harmonic and rhythmic
structures have to be incorporated into the modelling and as a result the
complexity is monumentally higher then \ac{MPE} approaches.
\cite{MIR-recent-dev:Schedl}

Readers interested in a comparison of the performance of different approaches
are referred to the Multiple Fundamental Frequency Estimation and Tracking task
of the annual \ac{MIREX}
(\url{http://www.music-ir.org/mirex}).


\subsection{Non-negative Matrix Factorization}

A large subset of transcription systems employ methods stemming from spectrogram
factorization techniques, which exploit the redundancies found in music
spectrograms. \ac{NMF} was first proposed by Lee
and Seung \cite{nmf1999:Seung}


Starting with a non-negative $M$ by $N$ matrix \textbf{X} the goal of \ac{NMF} is to
approximate it as a product of two non-negative matrics $ \textbf{W}_{M \times
    R}$ and $\textbf{H}_{R \times N}$, where $R \leq M$ such the cost function is
minimized :

\begin{equation}
  C = \;\mid \textbf{X} - \textbf{W}\cdot \textbf{H} \mid _ {F}
\end{equation}

where $\mid . \mid_{F}$ is the Frobenius norm. This is actually equivalent to
Gradient Descent based minimization of divergence. \cite{nmfamt2003:Smaragdis}
There are a number of algorithms for finding the appropriate values of
\textbf{W} and \textbf{H}. For example, the generalized Kullback-Leibler
divergence between \textbf{X} and $\textbf{W}\cdot \textbf{H}$ is non-increasing
under the following updates and guarantees the nonnegativity of both \textbf{W}
and \textbf{H} :

\begin{equation}
  \begin{split}
    H \Leftarrow H \odot \frac{W^{T}\frac{X}{WH}}{W^{T}J}
  \end{split}
  \text{\;\;and\;\;}
  \begin{split}
    W \Leftarrow W \odot \frac{\frac{X}{WH}H^{T}}{JH^{T}}
  \end{split}
  \label{review:eq:gradient_rules}
\end{equation}

where the $\odot$ operator denotes pointwise multiplication, $J \in
  \mathbb{R}^{M \times N}$ denotes the matrix of ones, and the divison is
pointwise. \cite{amt2019:Benetos}

In the context of time-frequency representations and \ac{AMT} both unknown matrices
have an intuitive interpretation. \textbf{X} in the most basic cases in
time-frequency analysis is a \ac{STFT} of the audio signal. \textbf{W} encodes the
spectral profiles of the \emph{R} components and is commonly referred to as the
dictionary matrix. \textbf{H} encodes the temporal activity of the each of those
components and named the activation matrix.

There are two classes of \ac{NMF} approches that fall into supervised and
unsupervised approaches. In supervised approaches the dictionary matrix is
prextracted. For explanatory purposes, one can imagine the applicaiton of such
an \ac{NMF} \ac{AMT} system. To compile the dictionary matrix a recording of each note
played in isolation is recorded and concatenated. Thus each component can be
thought of corresponding to individual pitches with their associated harmonic
profiles. The \ac{NMF}-based decomposition is then performed by applying the update
rules in \autoref{review:eq:gradient_rules} to find \textbf{H} to minimize the
cost function.

The unsupervised approach involves hyperparameter tuning to discover the optimal
value for the number of components. This can be achieved by grid search methods
cv-fold testing by splitting up the audio signal into smaller segments. Both
approaches are widely used and there have been many studies based on improving
performance and accuracy. For a comprehensive overview of a number of these
techniques refer to \cite{f0estimation2006:Cheveigne,Christensen:2009,
  spmmt:Klapuri}.

State of the art applications of \ac{NMF} for polyphonic \ac{AMT} include work were
sparseness constraints were added into the \ac{NMF} update rules, in an effort to
find meaningful transcriptions. \cite{NMF-sparsity:2006} Another approach was
based on incorporating harmonicity constraints in the \ac{NMF} model, resulting in
two algorithms: harmonic and inharmonic \ac{NMF}. \cite{inharm-harm-NMF:Vincent2010}
Additionally the model constrains each basis spectrum to be expressed as a
weighted sum of narrowband spectra, in order to preserve a smooth spectral
envelope. The inharmonic version of the algorithm is also able to support
deviations from perfect harmonicity and standard tuning. Also, another approach
proposed a Bayesian framework for \ac{NMF}, which considers each pitch as a model of
Gaussian components in harmonic positions. \cite{bayesian-NMF:Bertin2007}
Spectral smoothness constraints are incorporated into the likelihood function,
and for parameter estimation the space alternating generalised EM algorithm
is employed.

More recently, one approach proposed an algorithm for \ac{MPE} and
beat structure analysis. The \ac{NMF} objective function is constrained using
information from the rhythmic structure of the recording, which helps improve
transcription accuracy in highly repetitive recordings.
\cite{beat-NMF:Ochiai2012}


\subsection{Neural Networks}

\ac{NN} are systems that are vaguely inspired by biological neural
networks. They are based on a collection of connected units or nodes called
artifical neurons. They are able to learn non-linear functions from input to
output via an optimization algorithm. The goal of a network is to learn the
weights $w_{ij}$ by minimizing the cost function with respect to the training
data. \cite{DeepLearning2016}

\begin{figure}[!ht]
  \centering
  \includegraphics[width=.7\columnwidth]{\Pic{png}{neural-network}}
  \caption[Structure of a \ac{NN}]{An example of a feed forward architecture neural network appropriate for classification tasks. \cite{ISMIR-tut:Benetos}}
  \label{review:fig:neural-network}
\end{figure}


\ac{NN}s have a number of advantages over traditional machine learning algorithms.
One of the main advantages is the removal of the need for feature extraction
which is important in unstructured types of data like images or sound. The type
of network architecture that will be discussed in this paper is known as a feed
forward neural network. Deep networks partially replace the need for feature
engineering. The deeper layers in the network, model increasingly more abstract
and intricate features. In the context of music transcription, the layers closer
to the input might model individual notes, whilst deeper layers, model features
such as chords and harmonic progressions depending on the type of network used.


\begin{figure}[!b]
  \centering
  \includegraphics[width=.6\linewidth]{\Pic{jpg}{gradient-descent}}
  \caption[Gradient Descent]{An intuitive graphical understanding of gradient descent optimization}
  \label{review:fig:grad-descent}
\end{figure}

One extremely important concept that is pivotal for understanding the
optimization of neural networks is gradient descent and back propogation.
Backpropagation is an iterative optimization process used to train models. The
goal is to find the lowest point of a multivariate loss function by
incrementally updating each weight in the network by the product of the gradient
and the learning rate. \autoref{review:algo:grad} shows the steps that are used
in gradient descent optimization in one epoch. An epoch refers to a complete
pass through the training data. In other words all the samples have been passed
through the model for training Through backpropagation, the loss or difference
between the predicted value and the actual value, is transferred from one layer
to another and the weights are modified so that the loss is minimized.

\begin{algorithm}[!t] \AlgoFontSize \DontPrintSemicolon \KwData{Training dataset
    $X$} \KwResult{Optimal weights that minimize loss function $\theta$}
  \BlankLine

  \SetKwFunction{fgrad}{Calculate Gradient}
  \SetKwFunction{fDescent}{Gradient
    Based Optimization}

  \KwIn{query weight $\theta_{j}$}
  \KwOut{Gradient $\nabla$}
  \BlankLine

  \Func{\fgrad{$\theta_{j}$}}{
  \vspace{.5em}
  $r_{j} \longleftarrow \text{Slope of the loss function}$\;
  \vspace{.5em}
  $\alpha_{j} \longleftarrow \text{Slope of the activation function}$\;
  \vspace{.5em}
  $\gamma_{j}  \longleftarrow \text{Value of the neuron that feeds into our
      weight}$\;
  \vspace{.5em}
  $\nabla \longleftarrow \gamma_{j}~r_{j}~\alpha_{j}$ \;
  \vspace{.5em}
  \Return $\nabla$ \;}

  \BlankLine

  \KwIn{Initialized weights $\theta$, training Data $X$, learning rate $L$}
  \KwOut{Optimized weights $\theta$}
  \BlankLine

  \Proc{\fDescent{$\theta$, $X$, $L$}}{
    \vspace{.5em}
    \For{$X_{j} \in X$}{
      \vspace{.5em}
      Perform forward propagation with $X_{j}$ to make a prediction \;
      \vspace{.5em}
      Use this prediction in back propogation to update weights
      \;
      \vspace{.5em}
      $\nabla \longleftarrow$ \fgrad{$\theta_{j}$} \;
      \vspace{.5em}
      $\theta_{j} \longleftarrow \theta_{j} - L \times \nabla$} \Return
    $\theta$}

  \caption{Gradient descent optimization for one epoch with batch size equal to entire dataset}
  \label{review:algo:grad}
\end{algorithm}

One other crucial concept in \ac{NN}s and machine learning in general is the concept
of overfitting. Typically, datasets are seperated into test and train sets. The
model is exposed to the training data and is then used to predict on the test
data. When the test accuracy of the model starts to decrease and the training
accuracy further increases this is known as overfitting. This is because the
model is no longer capturing the underlying relationships in the data but is
rather overaccommodating to the subtleties in the training data. The end goal of
the network should be to predict on any unseen dataset that it has not been
exposed to and perform accurately.

There are a number of important parameters which have to be tuned in a neural
network through a process known as hyperparameter tuning. This is typically done
using a grid-search algorithm to find a set of parameters which optimizes the
cost function. \cite{LiDL2019} Some of these hyperparamters important for tuning
network performance are given below :

\subsubsection{Important Features in \ac{NN}s}
Several aspects of \ac{NN}s require further explanation as they are not obvious.
These parameters play a pivotal role in the performance and accuracy of the
network. This section will briefly discuss each of these parameters and certain
trade offs that they present.

\begin{enumerate}
  \item \emph{learning rate} - the rate at which the weights are updated in the
        optimization process. If the learning rate is too high the model may
        fail to converge in optimization.
  \item \emph{activation functions} - mathematical equations that determine the
        output of a node in network. They determine whether it should be
        activated or not based on the input to the node. A number of common
        activation functions that are used in different types of problems
        include : ReLU (Rectified Linear Unit), hyperbolic tangent, softmax
        and sigmoid activations. Each activation function is employed depending
        on the type of problem and corresponding optimizer and loss function
        used.

  \item \emph{optimizer} - the type of optimization algorithm employed to train
        the model and update the weights of the model. One of the most well
        known optimization algorithms is called gradient descent which has a
        number of variants such as stochastic and mini-batch gradient descent.
        Some optimization algorithms are more appropriate for certain types of
        problems such as regression or classification problems.
  \item \emph{epoch} - when an entire dataset is passed forward and backward
        through the \ac{NN} only once. Typically setting a higher number
        of epochs will lead to higher accuracy in the training set but their is
        a risk of overfitting.
  \item \emph{batch size} - total number of training samples present in a single
        portion of the dataset that is is used to update the weights in
        backpropagation. There is typically a tradeoff with batchsize,
        computational efficiency and accuracy. A smaller batch size requires
        more time for training but is generally more accurate. A common batch
        size that is used is 32 which is referred to as a mini-batch.
  \item \emph{loss function} - the loss function is used in backpropagation to
        update the weights so as to increase the prediction accuracy of the
        model. They are mathematical functions that measure the difference
        between the predicted output and the actual output. Some common loss
        functions include mean squared error, hinge, binary crossentropy (See
        \autoref{review:fig:log-loss}) and the kullback leibler divergence.

        \begin{figure}[!b]
          \centering
          \includegraphics[width=.8\columnwidth]{\Pic{png}{log_loss}}
          \caption[Binary cross entropy]{Binary cross entropy function that is useful in
            classification tasks used in conjuction with a softmax activation
            function on the output}
          \label{review:fig:log-loss}
        \end{figure}

\end{enumerate}

In recent years \ac{NN}s have had a considerable impact on the problem of music
transcription and on music signal processing in general. However, compared to
other fields progress on \ac{NN}s for music transcription has been slower due to a
lack of annotated data with labels which is essential for training the models
appropriately. \cite{end-to-end-transcription2017:Carvalho}

The current state of the art method for piano transcription was proposed in
research completed by Google Brain \cite{google2018:Elsen}. This approach
combines two networks, one which detects onsets and one which finds note
lengths. The output from the note onset network is used to inform the second
network calculating note lengths.

Despite the appeal of NNs and the promises they hold they are often still
outperformed by \ac{NMF}-based methods for a number of reasons :
\begin{enumerate}
  \item Lack of annotated labelled datasets - \ac{NN}s rely on data to be
        effective. There are only a a small number of annotated datasets which
        in themselves are restricted to certain types of instruments and genres
        of music. \cite{ground-truths:Su}
  \item Adaptablity to new conditions - there are currently no methods to
        retrain or adapt an \ac{NMF}-based \ac{AMT} systems on only a few seconds of audio.
        As such \ac{NMF}-based systems can perform considerably better with less data
        and are easier to adapt.
\end{enumerate}

\section{Summary}

In general there are drawbacks and advantages to both types of approaches
outlined in this chapter. \ac{NN}s can be more effective in context dependent
environments whilst \ac{NMF}s show more adaptablity. Currently there are no methods
which are favoured in all situations.

\autoref{ch:system} will introduce an \ac{AMT} system which uses \ac{NMF} to transcribe
single note melodies and discuss the system architecture. \autoref{ch:system}
will also discuss how to apply a \ac{NN} to musical data obtained from
\citeyear{thickstun2018invariances} MusicNet database.