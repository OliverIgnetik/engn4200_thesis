\SetPicSubDir{ch-Review} \SetExpSubDir{ch-Review}

\chapter{Literature Review}
\label{ch:review}
\vspace{2em}

\section{Signal Processing Techniques}
\subsection{Sampling Theorem}

The sampling theorem is a consequence of digitizing analogue signals. Sampling
an analogue signal stores quantized values of the amplitude of a continuous
signal at regular intervals determined by the sampling rate.

The sampling theorem says that to avoid higher frequency components aliasing as
lower frequencies components the following must be satisfied. Considering a
sampling frequency $F_{s}$ and Nyquist frequency $F_{N}$.

\begin{equation}
  F_{s} > 2\cdot F_{N}
\end{equation}

Where B is the highest frequency expected in the signal. Frequently a sampling
rate of 44.1 kHz is used in audio recording because the range of human hearing
is from 20-20kHz.


\subsection{Fourier Transform}
\subsection{Short-Time Fourier Transform}

As in other audio-related applications, the most popular tool for describing the
time-varying energy across different frequency bands is the short-time Fourier
Transform (STFT), which, when visualized as its magnitude, is known as the
spectrogram.

Formally, let $x$ be a discrete-time signal obtained by uniform sampling a
waveform at a sampling rate $F_{s}$ Hz. Using a N-point tapered window $w$ (eg.
Hamming $w[n] = 0.5-0.46\cdot cos(\frac{2\pi n}{N})$ for
$n\in\left[0,N-1\right]$) and an overlap of half a window length we obtain the
STFT.

\begin{equation}
  X [m,k] = \sum_{n=0}^{N-1}w[n]\cdot x[n + m\cdot\frac{N}{2}]\cdot exp\{-j\frac{2\pi k n }{N}\}
\end{equation}

With $m\in\left[0,T-1\right]$ and $k\in\left[0,K-1\right]$. Here, $T$ determines
the number of frames , $K = \frac{N}{2}$ is the index of the last unique
frequency value as dictated by the Sampling Theorem. Thus $X[m,k]$ corresponds :

\begin{align}
  f_{coeff}(k) & = \frac{k}{N} \cdot F_{s} \;\; Hz \\
  t_{frame}(m) & = t \cdot \frac{N}{2F_{s}} \;\; s
\end{align}

$X[m,k]$ is complex-valued, with the phase depending on the alignment of each
short-time analysis window. Often it is only the amplitude $\mid X[m,k] \mid$
that is used.

\subsubsection{Log-Frequency Spectrogram}

Note that the Fourier coefficients of $X[m,k]$ are linearly spaced on the
frequency axis. Using suitable binning strategies, various approaches switch
over to a logarithmically spaced frequency axis, by using mel-frequency bands or
pitch bands as seen in \autoref{review:fig:log-STFT-example}. Keeping the linear
frequency axis puts greater emphasis on the high-frequency regions of the
signal, thus accentuating the aforementioned noise bursts visible as
high-frequency content. One simple yet important step often applied in the
processing of music signals, is referred to as logarithmic compression. Such a
compression not only accounts for the logarithmic nature that describes how
humans perceive sound but also balances out the dynamic range of the signal.
\cite{spma2011:Klapuri}

\begin{figure}[!ht]
  \centering
  \includegraphics[width=.6\linewidth]{\Pic{png}{log-stft-example}}
  \caption{STFT of a 10s excerpt from Blues in F - Bill Evans Trio recording }
  \label{review:fig:log-STFT-example}
\end{figure}

\section{State of the Art Methods}
\subsection{Non-negative Matrix Factorization}
\subsection{Neural Networks}
\lipsum[1]

\section{Summary}

Literature review Summary