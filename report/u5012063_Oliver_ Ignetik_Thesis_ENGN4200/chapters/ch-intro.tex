\SetPicSubDir{ch-Intro}

\chapter{Introduction}
\vspace{2em}

The capability of transcribing music audio into music notation is a fascinating example of human intelligence. It involves perception (analyzing complex auditory scenes), cognition (recognizing musical objects), knowledge representation (forming musical structures), and inference (testing alternative hypotheses). Automatic music transcription (AMT), i.e., the design of computational algorithms to convert acoustic music signals into some form of music notation, is a challenging task in signal processing and artificial intelligence. It comprises several subtasks, including multipitch estimation (MPE), onset and offset detection, instrument recognition, beat and rhythm tracking, interpretation of expressive timing and dynamics and score typesetting. 
\cite{amt2019:Benetos}

A successful AMT system would enable a broad range of interactions between people and music, including music education (e.g., through systems for automatic instrument tutoring), music creation (e.g., dictating improvised musical ideas and automatic music accompaniment), music production (e.g., music content visualization and intelligent content-based editing), music search (e.g., indexing and recommendation of music by melody, bass, rhythm, or chord progression), and musicology (e.g., analyzing jazz improvisations and other nonannotated music). As such, AMT is an enabling technology with clear potential for both economic and societal impact. \cite{amtfc2013:Benetos}

\section{Overview}

\subsection{Research Question}

Can incorporating higher level musical knowledge into existing AMT systems improve the perceived quality of the output for human listeners ? 

\subsection{Project Scope}

\lipsum[1]

\section{Thesis Synopsis}

The rest of this thesis is organized as follows. 
In \autoref{ch:review}, we conduct a literature review. We will review crucial Digital Signal Processing (DSP) techniques and expand upon these with methods unique to music DSP. Finally we will investigate relevant state of the art methods in music DSP. 

\autoref{ch:system} provides details on the system architecture of the AMT systems used in this research project. 
\autoref{ch:results} . 
We conclude the entire thesis as well as discuss further directions for future research in \autoref{ch:concl}.

\section{Resources}

You may go to \href{https://github.com/OliverIgnetik/engn4200_thesis}{my GitHub repository} or access the url  \url{https://github.com/OliverIgnetik/engn4200_thesis} directly. 
