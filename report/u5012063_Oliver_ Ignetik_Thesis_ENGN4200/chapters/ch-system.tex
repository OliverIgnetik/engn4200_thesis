\SetPicSubDir{ch-System}

\chapter{System Design}
\label{ch:system}
\vspace{2em}

The AMT problem can be divided into several subtasks, which include: multipitch
detection, note onset/offset detection, loudness estimation and quantisation,
instrument recognition, extraction of rhythmic information, and time
quantisation. The core problem in automatic transcription is the estimation of
concurrent pitches in a time frame, also called multiple-F0 or multi-pitch
detection. \cite{amtfc2013:Benetos}

\begin{figure}
    \centering
    \includegraphics[width=.5\linewidth]{\Pic{png}{model}}
    \vspace{\BeforeCaptionVSpace}
    \caption{Proposed general architecture of a music transcription system. \cite{amtfc2013:Benetos}}
    \label{system:fig:model}
\end{figure}

\section{Preliminaries}

This report will investigate how to use NMF in interpreting an audio recording and extracting
music information from this recording. There are two systems and two applications that will be
be presented in this section.

\begin{enumerate}
    \item NMF applied to single line melody - exploration of parameters used to fine tune model accuracy
          \begin{enumerate}
              \item MAPS Dataset - chromatic scale played on a Bechstein D 280 in a concert hall \cite{MAPS:Emiya}
              \item Type of architecture - Supervised NMF
          \end{enumerate}
    \item NN applied to a large music database
          \begin{enumerate}
              \item Dataset MusicNet recordings and active frame note labels \cite{thickstun2018invariances}
              \item Type of architecture - Feed Forward neural network with mutli-label classification
          \end{enumerate}
\end{enumerate}

\newpage
\section{Methodology}

\subsection{NMF}
The experiment for the NMF procedure is outlined in \autoref{system:algo:NMF}.
This process forms the basis of the investigation and the effect of hyperparameters such as
the window size and frequency resolution will be investigated.

\begin{algorithm}
    \AlgoFontSize
    \DontPrintSemicolon
    \KwData{Excerpt of audio recording $x$}
    \KwResult{Array of active notes pitch and onset times $y$}
    \BlankLine

    \KwIn{$x$ audio signal sampled at 44.1 kHz}
    \KwOut{$y$ predicted pitches}
    \BlankLine
    \SetKwFunction{NMFAMT}{NMF Pitch Detection}

    \Proc{\NMFAMT{$x$}}{
        \vspace{.5em}
        Find corresponding ground truth labels of audio excerpt from database \;
        Convert ground truth labels into DataFrame containing $OnsetTime$ and $MidiPitch$ \;
        $number\_of\_notes \longleftarrow length(ground\_truths)$
        $X = \sum_{n=0}^{N-1}  w[n] \cdot x[n+t+N/2]\cdot e^{-j2 \pi \frac{kn}{N}}$ \;
        Establish frequency resolution corresponding to a semitone \;
        Perform harmonic and percussive separation with Librosa function \;
        Perform NMF decomposition \;
        Make use of Peak picking algorithm on \;
        dictionary spectral templates to find f0 \;
        Convert f0 to MIDI pitch \;
        Use peak picking algorithm to find onset time for each activation profile \;
        $y \longleftarrow$ Merge onset times and midi pitches \;
    }
    \caption{NMF procedure for single note melody transcription}
    \label{system:algo:NMF}
\end{algorithm}

\newpage
\subsection{NN}
The experiment for the NN procedure is outlined in \autoref{system:algo:NN}.
This process forms the basis of the investigation and the effect of hyperparameters such as
loss function choice and number of layers will be investigated.

\begin{algorithm}
    \AlgoFontSize
    \DontPrintSemicolon
    \KwData{Dataset$X$}
    \KwResult{Predicted active notes in each frame $y$}
    \BlankLine

    \SetKwFunction{fNN}{NN transcription procedure}

    \KwIn{Dataset of audio excerpts $X$}
    \KwOut{Predicted active notes in each frame $y$}
    \BlankLine

    \Proc{\fNN{$X$}}{
    }
    \vspace{.5em}
    \caption{Procedure for NN investigation}
    \label{system:algo:NN}
\end{algorithm}

\newpage
\section{Evaluation}
\subsection{NMF}
MENTION TESTING AND ACCURACY
    [INSERT TABLE OF PARAMETERS TO BE TESTED]
\subsection{NN}
MENTION TESTING AND ACCURACY
    [INSERT TABLE OF PARAMETERS TO BE TESTED]

% \begin{figure}[!t] \centering \begin{minipage}[b]{.45\linewidth} \centering
%         \includegraphics[width=\linewidth]{\Pic{png}{model}}
%         \caption{placeholder} \label{system:fig:model2}
%     \end{minipage} \hspace*{2em} \begin{minipage}[b]{.45\linewidth} \centering
%         \includegraphics[width=\linewidth]{\Pic{png}{model}}
%         \caption{placeholder} \label{system:fig:model3}
%     \end{minipage} \end{figure}

\section{Summary}
